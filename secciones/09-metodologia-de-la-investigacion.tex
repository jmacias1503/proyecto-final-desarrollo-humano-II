\subsection{Enfoque de la investigación}
Dado que se busca comprender y comprobar la hipótesis previamente establecida,
así como los objetivos trazados, el presente trabajo será elaborado bajo el
planteamiento metodológico del enfoque mixto.
\subsection{Tipo de Estudio Observacional}
Un estudio observacional es un tipo de estudio en el que el investigador no interviene en las
variables de estudio. En otras palabras, el investigador se limita a observar los valores de
las variables estudiadas y hacer mediciones. A continuación, te proporciono más detalles
sobre los estudios observacionales
\subsection{El enfoque mixto}
Los estudios con enfoque mixto representan un 
\begin{quote}
conjunto de procesos sistemáticos , empíricos y críticos de investigación e
implican la recolección y el análisis de datos cuantitativos y cualitativos, así
como su integración y discusión conjunta, para realizar inferencias producto de
toda la información recabada y lograr un mayor entendimiento del fenómeno bajo
estudio \parencite{unitec}
\end{quote}
Este enfoque permite abordar preguntas de investigación desde diferentes
perspectivas y enriquecer los resultados obtenidos.
\subsubsection{Variables de estudio}
\begin{itemize}
\item Variables independientes: inteligencia intrapersonal, medida a través de pruebas estandarizadas que evalúan las diferentes dimensiones intrapersonales.
\item Variables dependientes: Desarrollo profesional, medido mediante indicadores como logros profesionales o satisfacción laboral.
\end{itemize}
\subsection{Encuestas}
Realizamos encuestas con el fin de recopilar información relevante y necesaria para lograr
comprender y comprobar nuestra hipótesis y objetivos establecidos previamente. Las
encuestas se llevaron a cabo en persona. Se realiz\'o una encuesta a un grupo de estudiantes. A continuación, las preguntas que realizamos en las encuestas
\subsubsection{Datos demogr\'aficos}
\begin{itemize}
\item Edad
\item Nivel educativo
\item \'Area de estudio o especialidad
\end{itemize}
\subsubsection{Autoevaluaci\'on de inteligencia intrapersonal}
\begin{itemize}
\item  En una escala del 1 al 5, ¿cómo calificarías tu capacidad para conocerte a ti mismo/a?
\item  ¿Consideras que gestionas eficazmente tus emociones en el ámbito profesional? 
\item  ¿Te consideras una persona motivada internamente en tu carrera profesional? 
\end{itemize}
\subsubsection{Percepci\'on sombre la influencia de la inteligencia
intrapersonal en el desarrollo profesional}
\begin{itemize}
\item ¿Crees que tener un buen autoconocimiento influye en la toma de decisiones profesionales?
\item ¿Cómo crees que la gestión emocional impacta en el desempeño laboral?
\item ¿Consideras que la motivación interna es clave para el éxito en la carrera profesional?
\item ¿Qué habilidades asociadas a la inteligencia intrapersonal crees que son más importantes para el desarrollo profesional? (Por ejemplo: autorreflexión, autodisciplina, resiliencia)
\end{itemize}
\subsubsection{Eperiencias y estrategias personales}
\begin{itemize}
\item ¿Has experimentado situaciones donde la inteligencia intrapersonal haya sido crucial para resolver problemas en el trabajo/estudios?
\item ¿Qué estrategias utilizas para mejorar tu inteligencia intrapersonal en el ámbito laboral?
\end{itemize}
\subsubsection{Opini\'on sobre la importancia de la inteligencia intrapersonal
en el entorno laboral}
\begin{itemize}
\item ¿Crees que la inteligencia intrapersonal debería ser más valorada en el ámbito laboral? 
\item ¿Qué medidas crees que podrían implementarse para fomentar el desarrollo de la inteligencia intrapersonal en el lugar de trabajo?
\end{itemize}
