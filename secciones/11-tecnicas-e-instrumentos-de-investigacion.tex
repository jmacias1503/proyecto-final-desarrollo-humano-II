Para esta investigación, el uso de técnicas e instrumentos adecuados es importante para la obtención de datos precisos y que sean relevantes. Estos métodos nos permiten explorar, analizar y comprender diferentes contextos de una manera estructurada y sistemática. Entre los diversos instrumentos de investigación que se pueden utilizar, decidimos utilizar las encuestas gracias a su eficacia para recopilar información de grandes grupos de personas en un tiempo relativamente corto. 

Esta encuesta, nos permitió obtener percepciones, opiniones y comportamientos de los participantes respecto a temas específicos. Las preguntas formuladas no sólo recogen información cuantitativa sino también cualitativa, proporcionando una visión integral del tema estudiado. En particular, la encuesta fue dirigida a los estudiantes del grupo 31 de la carrera de Ingeniería en Software. Este grupo específico de encuestados ofrece una perspectiva única sobre cómo los futuros profesionales en el ámbito tecnológico pueden llegar a percibir y gestionar su inteligencia intrapersonal. 

A continuación se presentan las preguntas que formaron parte de la encuesta realizada y por qué de dichas preguntas.

\begin{enumerate}
\item Edad
Explicación: La edad puede influir en el nivel de desarrollo de la inteligencia intrapersonal, ya que las experiencias y la madurez suelen aumentar con el tiempo, afectando la autoevaluación y gestión emocional.
\item En una escala del 1 al 5 ?`c\'omo calificar\'ias tu capacidad para conocerte
a ti mismo(a)?\\
Explicación: Esta pregunta ayuda a medir directamente la autoevaluación de la inteligencia intrapersonal, proporcionando una visión del nivel de autoconocimiento del encuestado.
\item ?`Consideras que podr\'ias gestionar eficazmente tus emociones en el \'ambito
profesional?\\
Explicación: La gestión emocional es una componente clave de la inteligencia intrapersonal, y esta pregunta evalúa cómo los encuestados perciben su capacidad para manejar emociones en un contexto profesional.
\item ?`Te consideras una persona motivada internamente en tu carrera profesional?\\
Explicación: La motivación interna es otro aspecto esencial de la inteligencia intrapersonal. Evaluar esto ayuda a entender si los individuos se sienten impulsados por factores internos en su desarrollo profesional.
\item ?`Crees que tener un buen autoconocimiento influye en la toma de decisiones
Explicación: Esta pregunta explora la percepción de los encuestados sobre la relevancia del autoconocimiento en la toma de decisiones, un aspecto crítico del desarrollo profesional.
profesionales?\\
\item ?`C\'omo crees que la gesti\'on emocional impacta en el desempe\~no laboral?\\
Explicación: Evaluar la percepción sobre la gestión emocional permite entender cómo los alumnos creen que sus habilidades intrapersonales podrían afectar su rendimiento en el trabajo.
\item ?`Consideras que la motivaci\'on interna es clave para el \'exito en la carrera profesional?\\
Explicación: Esta pregunta investiga la importancia a la motivación interna en el éxito profesional, una faceta de la inteligencia intrapersonal que puede ser crucial para el desarrollo profesional.
\item ?`Has experimentado situaciones donde la inteligencia intrapersonal haya sido
crucial para resolver problemas en el trabajo/estudios?\\
Explicación: Esta pregunta busca ejemplos prácticos donde la inteligencia intrapersonal ha sido aplicada, proporcionando evidencia real de su importancia y utilidad en situaciones específicas.
\item ?`Crees que la inteligencia intrapersonal deber\'ia ser m\'as valorada en el
\'ambito laboral?\\
Explicación: Esta pregunta aborda la importancia de la inteligencia intrapersonal en el lugar de trabajo, lo cual puede influir en políticas y prácticas dentro de una área laboral.
\begin{table}
	\caption{Pregunta 1\label{tab:pregunta1}}
\begin{tabular}{cc}
\hline
Edad & Cantidad Personas\\
\hline
18 & 10\\
19 & 6\\
20 & 1\\
21 & 1\\
22 & 2\\
23 & 1\\
\hline
\end{tabular}
\end{table}
\begin{table}
	\caption{Pregunta \thetable\label{pregunta2}}
\begin{tabular}{cc}
\hline
Respuestas & Cantidad Personas\\
\hline
1 & 2\\
2 & 3\\
3 & 6\\
4 & 8\\
5 & 2\\
\hline
\end{tabular}
\end{table}
\begin{table}
	\caption{Pregunta \thetable\label{pregunta3}}
\begin{tabular}{cc}
\hline
Respuestas & Cantidad Personas\\
\hline
S\'i & 13\\
No & 8\\
\hline
\end{tabular}
\end{table}
\begin{table}
	\caption{Pregunta \thetable\label{pregunta4}}
\begin{tabular}{cc}
\hline
Respuestas & Cantidad Personas\\
\hline
S\'i & 10\\
No & 11\\
\hline
\end{tabular}
\end{table}

\begin{table}
	\caption{Pregunta \thetable\label{pregunta5}}
\begin{tabular}{cc}
\hline
Respuestas & Cantidad Personas\\
\hline
S\'i & 21\\
No & 0\\
\hline
\end{tabular}
\end{table}
En la pregunta 6, se dieron varias respuestas entre las que más se repiten están
\begin{itemize}
\item Aumenta la productividad diaria. 
\item Mejora la comunicación interna. 
\item Reduce el estrés laboral. 
\item Fomenta un ambiente positivo. 
\item Incrementa la motivación personal.
\end{itemize}
\begin{table}
	\caption{Pregunta \thetable\label{pregunta7}}
\begin{tabular}{cc}
\hline
Respuestas & Cantidad Personas\\
\hline
S\'i & 21\\
No & 0\\
\hline
\end{tabular}
\end{table}
\begin{table}
	\caption{Pregunta \thetable\label{pregunta8}}
\begin{tabular}{cc}
\hline
Respuestas & Cantidad Personas\\
\hline
S\'i & 15\\
No & 6\\
\hline
\end{tabular}
\end{table}
\begin{table}
	\caption{Pregunta \thetable\label{pregunta9}}
\begin{tabular}{cc}
\hline
Respuestas & Cantidad Personas\\
\hline
S\'i & 17\\
No & 4\\
\hline
\end{tabular}
\end{table}
\end{enumerate}
