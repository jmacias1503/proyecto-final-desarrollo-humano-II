\section{Presentación de Resultados}

Por medio del trabajo de investigación se ha indagado y aplicado los conocimientos teóricos de los investigadores, además de las habilidades desarrolladas según los contenidos con respecto a la inteligencia múltiple intrapersonal y su relación con la conducta que genera a lo largo del desarrollo profesional, esto por medio del análisis de su impacto. 

Partimos de una introducción que recopila las ideas principales y contenido del proyecto, se planteó la problemática seleccionada y se generaron preguntas de apoyo para delimitar adecuadamente las ideas, todo esto manteniendo fijos los objetivos deseados y su aplicación, contando con los fundamentos y justificaciones convincentes y objetivos, de forma que se llegó a una hipótesis pertinente, la cual procedió a ser evaluada por un marco teórico bien estructurado y que pudo complementarse más adelante con una metodología que cerrara las ideas y las mantuviera en un enfoque oportuno. 

Por medio de las consideraciones previamente mencionadas, se analizó a una muestra por un instrumento de investigación definido previamente, la cual estaba conformada por los integrantes del grupo. Finalmente cabe destacar que en la investigación se recopila y analiza la información rescatada, pasando por un análisis e implementación con el objetivo principal y los apartados iniciales del proyecto. 

\section{Conclusión}

La realización de la investigación ha representado por parte de los integrantes el dominio de unidades previas, contando con su entendimiento y aplicación para alcanzar el objetivo en común. Pese a las distintas complicaciones que se puedan llegar a presentar, desde la falta de entendimiento hasta problemáticas en la comunicación, la importancia radica en la mejora de nosotros mismos como profesionistas y los cimientos con los que estamos construyendo nuestro futuro.
