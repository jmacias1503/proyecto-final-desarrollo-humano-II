\section{Planteamiento del Problema}
De primera instancia, se quiso buscar la correlaci\'on de esta inteligencia --
que se denota m\'as como una que trabaja solo con el interior del individuo, sin
embargo, se decidi\'o explorar en esta investigaci\'on el c\'omo es que puede
influir en lo externo. A la vez, con esto se busc\'o relacionar el por qu\'e del
\'exito de individuos catalogados como introvertidos, que normalmente llegan a
desarrollar de una mejor manera esta inteligencia que los individuos
clasificados como extrovertidos
\section{Preguntas de Investigaci\'on}
Para poder tener un mejor esquema de la situaci\'on con la inteligencia
intrapersonal, se decidi\'o seguir un margen de preguntas para poder investigar
su naturaleza
\begin{itemize}
\item ?`Por qu\'e puede influir esta inteligencia en lo externo?
\item ?`C\'omo puede afectar en el trabajo?
\item ?`Se puede desarrollar mejor esta inteligencia para poder tener un mayor
\'exito en el \'area laboral?
\item ?`A qu\'e medios se debe acudir para conocer mejor de manera intr\'inseca
	esta inteligencia?
\item ?`C\'omo se puede evaluar esta inteligencia?
\item ?`Qu\'e representan los resultados al evaluar esta inteligencia?
\item ?`C\'omo puede estar vinculada con la inteligencia emocional?
\end{itemize}
