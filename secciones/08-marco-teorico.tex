Todo esto empieza con Howard Garner quien en la década de los 90 presenta la
tesis sobre la existencia de las Inteligencias múltiples, esta teoría nace de
una manera muy peculiar debido que nació directamente de la psicología, pero la
mayoría de los psicólogos les desagradó y otros simplemente las ignoraron, sin
embargo, desde el sector educativo comenzó a poner más atención a esta teoría,
ahí es donde entra Howard Garner para desarrollar totalmente esta idea.\\
Estas inteligencias las define más bien como capacidades que tiene un ser humano
y se van desarrollando a lo largo de su vida. Esta teoría da a conocer que los
conocimientos que adquirimos son de manera intuitiva y no es necesario tener una
brillantez académica para desarrollarla u obtenerla por lo que cada inteligencia
expresa una capacidad que opera de acuerdo a sus propios sistemas, bases o
reglas biológicas, puede haber gente que desarrolle un aspecto mejor que otra de
una manera muy natural, pero eso no significa que la otra persona no sea
inteligente si no que no ha desarrollado la habilidad a comparar pero puede que
esta persona la tenga en otro aspecto o situación. Por esto mismo Gardner da a
conocer que todos tenemos siete inteligencias modulares esto quiere decir que
cada una es autónoma e independiente de las otras, se combinan en el individuo y
el aspecto social, pero entre ellas no se influyen.\\
Las siete inteligencias que present\'o son las siguientes:
\begin{itemize}
\item Lingüístico-verbal.
\item Lógico-matemática.
\item Musical.
\item Espacial.
\item Científico-corporal.
\item Interpersonal.
\item Intrapersonal.
\item Naturalística.
\end{itemize}
Estas son las siete principales, sin embargo en los inicios de los 2000´s,
empezó a investigar dos probables nuevas inteligencias, pero se mantienen
hipotéticos, las cuales son la inteligencia moral que se define como la
habilidad de algunas personas para discernir lo que comúnmente se conoce como el
bien y el mal, y la otra es la inteligencia existencial que esta es un poco más
compleja ya que es la que quiere señalar una sensibilidad por la existencia del
ser humano reflexionando mucho las trascendencia humana.\\
Sin embargo, en este trabajo nos enfocaremos en uno en particular, que es la
inteligencia interpersonal que se define como “la capacidad de entender a los
demás e interactuar eficazmente con ellos. Incluye la sensibilidad a expresiones
faciales, la voz, los gestos y posturas y la habilidad para responder”. En lo
que consiste es el relacionarse y ser empáticos con otras personas, reconocer y
apreciar opiniones, formas de pensar, gustos, etc.; a pesar de etas diferencias
es mantener una relación amable y de buena comunicación siendo respetuoso con
los demás, normalmente se ve este tipo de actitudes en personas que tienen un
oficio que se enfoca mucho al ojo público, como lo son actores, personas de la
farándula, políticos, o gente que represente un grupo social.\\
El enfoque que le daremos a esta inteligencia será dentro del ambiente laboral,
como esta capacidad puede afectar en un ambiente en el cual estas en constante
contacto con personas que tienen distintas personalidades tratando de tener un
gran ambiente. La importancia de tener afinada esta cualidad es mantener un
canal de comunicación efectivo mejorando el trabajo en equipo, capacidad de
resolver conflictos también es muy importante en estos ambientes mostrando
liderazgo y una mentalidad de querer superarse. Aquí entra otro tema que servirá
de mucha importancia, que es la inteligencia emocional quien influye demasiado
en la manera con la que actúas socialmente con las personas.\\
Según Peter Salovey y John Mayer definen la inteligencia emocional como
“la capacidad de controlar y regular los sentimientos de uno mismo y de los
demás y utilizarlos como guía del pensamiento y de la acción”; da a entender que
estos hay que usarlos de manera que ayuden a motivarnos, y controlarlos a pesar
de el estado de nuestro ánimo. Años después Daniel Goleman (2007) dice que “la
inteligencia emocional está relacionada con un conjunto de habilidades que se
basan en la capacidad de reconocer los sentimientos propios y ajenos para que
sirvan de guía al pensamiento y a la acción”; al analizar estas definiciones
coinciden en algunas cosas que son la capacidad de aprender, comprender y
resolver problemas.\\
En el año de 1997 Mayer y Salovey presentaron un modelo donde se dividió a la
inteligencia emocional en 4 habilidades que se presentaran a continuación:
\begin{itemize}
\item Percepción Emocional: Esta identifica y reconoce sentimientos propios y
ajenos, prestar atención a las señales corporales o emocionales que presentan
las personas al interactuar con ellas. 
\item Asimilación emocional: Se tienen en cuenta las emociones de la otra
persona cuando se necesita razonar o solucionar algún problema, priorizando los
procesos cognitivos y centrándose en lo importante.
\item Comprensión Emocional: Es la habilidad de comprender las emociones en
función de relaciones de una manera compleja, además de reconocer posibles
transformaciones de las mismas emociones.
\item Regulación Emocional: Es la capacidad de estar abierto a sentimientos
negativos y positivas, reflexionar y gestionar las emociones de uno mismo y los
demás.
\end{itemize}
Estas cuatro divisiones si los reflexionan se aplicarían de una manera muy
importante en el ambiente laboral, percibiendo las emociones de los compañeros,
así como asimilar los problemas de organización, comunicación que se suelen dar
y manejarlos emocionalmente de una forma correcta y armoniosa, haciendo un clima
laboral mas saludable, casi como una familia. Sin embargo, hay seis puntos que
hay que tomar muy en cuenta para llevarlas a cabo en un buen ambiente laboral
las cuales se presentan a continuación.
\subsection{Prevención de riesgos laborales}
Esta mejora los entornos psicosociales que llevan a una mejora en la empresa,
siendo flexibles en algunas reglas o factores situacionales que se pudieran dar
durante el horario laboral, por ejemplo, el poner horarios flexibles, ofertar
actividades que combatan el estrés en la oficina, autonomía en la organización
del trabajo entre muchas más.
\subsection{Motivación}
Según Oropeza (2014), los individuos necesitan sentirse valorados y que sus
esfuerzos en la empresa sean reconocidos, ya que es una de las maneras por los
que la motivación en éstos aumentará, dado que se sentirán a gusto y realizados.
\subsection{Negociación}
Las emociones tienen un papel muy importante en la negociación, ya que muchos
autores establecen que los negociadores que logren una mayor comprensión de las
experiencias y expresiones emocionales estarán mejor preparados para la
negociación. Así pues, los negociadores con una alta IE conseguirán información
de una forma más rápida, tomarán decisiones más objetivas, mostrarán las
emociones adecuadas e inducirán la emoción deseada en el oponente. (López, 2013)
.
\subsection{Gestión y resolución del conflicto}
Hay conflictos que en cuanto aparecen deben desaparecer, en cambio, en
determinadas circunstancias, es necesario incentivarlo para que el resultado del
equipo tenga una gran calidad. Resolver un conflicto significa reducirlo o
eliminarlo; pero gestionarlo implica diseñar estrategias para incrementar la
efectividad de la organización. (Munduate y Medina, 2015).
\subsection{Estrés}
Es una condición mental y física que puede llegar cuando se presiona con mucha
carga a los empleados o trabajadores de una empresa, pero no es todo negativo ya
que si aparece en una dosis adecuada da como una motivación para asumir y
realizar determinadas tareas.
\subsection{Liderazgo}
El liderazgo es uno de los roles que más trabajo emocional implica, debido a que
tienen que estar en un contacto directo con sus colaboradores. Por tanto, la
capacidad de un líder para llevar a un grupo a conseguir unos objetivos es un
tema que tiene una gran importancia como proceso grupal en una organización.
(López, 2013).
