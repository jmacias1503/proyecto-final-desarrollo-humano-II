El presente trabajo se elaboró bajo el planteamiento de un enfoque metodológico de un enfoque
mixto. Para llevar a cabo esta investigación, se emplearon métodos cualitativos y cuantitativos
realizando un cuestionario basado en 9 preguntas , este cuestionario se les realizó a 21 estudiantes
de segundo semestre de la carrera de ingeniería en software dentro de la Universidad Autónoma de
Querétaro Facultad de informática
Este enfoque permite abordar preguntas de investigación desde diferentes perspectivas y enriquecer
los resultados obtenidos.

Las preguntas de la encuesta se enfocaron en aspectos como:
\begin{itemize}
\item Identificación del autoconocimiento y autorreflexión de los sujetos estudiados.
\item Influencia de la conducta en ámbitos profesionales.
\item Motivación intrínseca y metas profesionales.
\item Gestión y resolución de conflictos.
\item Prevención de riesgos laborales.
\item Percepción, comprensión y asimilación emociona
\end{itemize}
El cuestionario fue profundizado en experiencias más personales en base a algunas percepciones
subjetivas sobre cómo la inteligencia intrapersonal ha influido en su ámbito laboral.
Los resultados mostraron que la mayoría de los estudiantes se consideran con una alta inteligencia
intrapersonal ya que en un ámbito tecnológico pueden llegar a percibir y gestionar su personalidad,
sus capacidades y sus emociones de una manera eficaz

En el cuestionario se incluyeron las siguientes preguntas
\begin{enumerate}
\item Edad
\item En una escala del 1 al 5, ¿cómo calificarías tu capacidad para conocerte a ti mismo/a?
\item ¿Consideras que podrías gestionar eficazmente tus emociones en el ámbito profesional?
\item ¿Te consideras una persona motivada internamente en tu carrera profesional?
\item ¿Crees que tener un buen autoconocimiento influye en la toma de decisiones profesionales?
\item ¿Cómo crees que la gestión emocional impacta en el desempeño laboral?
\item ¿Consideras que la motivación interna es clave para el éxito en la carrera profesional?
\item ¿Has experimentado situaciones donde la inteligencia intrapersonal haya sido crucial para resolver problemas en el trabajo/estudios?
\item ¿Crees que la inteligencia intrapersonal debería ser más valorada en el ámbito laboral?
\end{enumerate}

\subsection{Resultados y Estad\'isticas}
En la pregunta uno se muestran las edades de los encuestados, la edad que predominó con más del
50\% fue la de 10 alumnos con 18 años, 6 alumnos con 19 años, 1 alumno con 20 años, 1 alumno con
21 años, 2 alumnos con 22 años con y 1 alumno con 23 años.\\
Para la pregunta dos hace un análisis en base a una puntuación del 1 (siento la calificación menor)
hasta el 5 (siendo la calificación mayor) y esta refleja que 8 alumnos consideran un 4 como su
capacidad de conocerse a sí mismo, 6 alumnos optan por una puntuación de 3, 3 alumnos con una
puntuación de 2, 2 alumnos con 1 punto y finalmente 2 alumnos con 5 puntos.\\
La pregunta tres se centra en poder saber la eficacia de los alumnos al gestionar sus emociones en el
ámbito laboral, en esta pregunta se pudo analizar que el 61.9 \% (13 alumnos)de los alumnos SI
pueden gestionar con eficacia, por otro lado, el 38.1 \% (8 alumnos) de los alumnos NO se creen
capaces.\\
La pregunta cuatro muestra a los estudiantes que se muestran motivados internamente en tu carrera
profesional, de lo cual el 52.38\% (11 alumnos) de los alumnos NO cuentan con esa motivación,
mientras que el 47.6p2 \% (10 alumnos) SI tiene esa motivación.\\
La pregunta cinco hace referencia al saber de cada alumno por la necesidad del autoconocimiento
de su toma de decisiones profesionales, en este caso se obtuvo un resultado del 100 \% ya que 21
alumnos votaron que SI creen que al tener un buen autoconocimiento influye en la toma de
decisiones profesionales.\\
Dentro de la pregunta seis se quería saber si los alumnos tienen la noción de como la gestión
emocional impacta en el desempeño laboral y se reflejaron varias respuestas, algunas de las más
relevantes y repetitivas fueron:
\begin{itemize}
\item Aumenta la productividad diaria.
\item Mejora la comunicación interna.
\item Reduce el estrés laboral.
\item Fomenta un ambiente positivo.
\item Incrementa la motivación personal.
\end{itemize}
La pregunta siete se realizó con la finalidad de ver la consideración de cada alumno a cerca de la
motivación interna y si la consideran clave para el éxito en la carrera profesional, se obtuvo una
respuesta de un 100 \% por parte de los alumnos, ya que 21 de los 21 alumnos optan que la motivación
interna SI es clave para el éxito en la carrera profesional.\\
Para la pregunta ocho se basa en una pregunta más personal ya que al realizar esta pregunta se tiene
el propósito de saber si los alumnos han experimentado situaciones donde la inteligencia
intrapersonal haya sido crucial para resolver problemas en el trabajo/estudios, y se obtuvo como
resultado que: el 71.43\% (15 alumnos) SI han vivido está experiencia y el 28.57 \% (6 alumnos) NO la
han vivido.\\
En la pregunta nueve se considera el valor individual que le tiene a la inteligencia intrapersonal y si
consideran que se le debería de dar más valor dentro del ámbito laboral. En base a esta pregunta, se
pudo analizar que por mayoría, el 80.95\% (17 alumnos) de los alumnos SI creen que su valor debe
de ser mayor, mientras que el 19.05\% (4 alumnos) dice lo contrario. Como dato relevante,
El 85\% de los encuestados con alta inteligencia intrapersonal reportaron
habilidades superiores en la resolución de conflictos, en contraste con el 40\% de aquellos con baja
inteligencia intrapersonal.
La investigación empleó una técnica principal: encuesta. Las encuestas proporcionaron datos
cuantitativos sobre la población estudiada ya que midieron diversos aspectos de la inteligencia
intrapersonal basándose en un ámbito personal.
