Este artículo surge por dos motivos fundamentales. En primer lugar, busca
recrear un ambiente laboral en el aula con el objetivo de promover la
experiencia del trabajo en equipo. Nos interesa observar cómo el compromiso, la
responsabilidad, la gestión del tiempo y el esfuerzo conjunto pueden llevarnos a
alcanzar objetivos más ambiciosos que cuando trabajamos de manera individual o
en equipos desorganizados.\\
En segundo lugar, este estudio pretende resaltar la importancia de la
introspección en el entorno laboral. A menudo se enfatiza la relevancia de
habilidades interpersonales como la comunicación y la expresión, pero rara vez
se aborda el valor de la introspección. Este artículo busca demostrar el impacto
significativo de la inteligencia intrapersonal en el desarrollo de las''soft
skill'' y su influencia positiva en el desempeño laboral.\\
Al explorar la interacción entre el trabajo en equipo y la introspección, este
artículo busca ofrecer una perspectiva más completa sobre cómo las habilidades
intrapersonales pueden complementar y potenciar las habilidades interpersonales
en un entorno laboral dinámico.\\
